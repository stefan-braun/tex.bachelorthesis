%!TEX root = ../document.tex
\usepackage[section=chapter,toc,acronym,style=long,nolist]{glossaries}

% \makenoidxglossaries
%\makeglossaries

\usepackage{imakeidx}
\usepackage{acronym}
% \makeindex


\renewcommand*{\indexpagestyle}{chapterpage}

%Further Settings
\setacronymstyle{footnote-sc}


% Acronym definitions
\newacronym{SOAP}{SOAP}{Simple Object Access Protocol}
\newacronym{CFML}{CFML}{ColdFusion Markup Language}
\newacronym{SQL}{SQL}{Structured Query Language}

% Term definitions
%TODO Definitions
\newglossaryentry{Backend}{
	name=Backend,
	description={Nicht das Frontend}
}
\newglossaryentry{SearchEngine}{
	name=Search Engine,
	description={Zum Suchen}
}
\newglossaryentry{ReEvent}{
	name=ReEvent,
	description={Veranstaltungsverwaltungssystem auf Basis des PHP-Frameworks Flow, Nachfolger von INsite als Veranstaltungsverwaltung der Stadt Ingolstadt.}
}
\newglossaryentry{INsite}{
	name=INsite,
	description={Veranstaltungsverwaltung der Stadt Ingolstadt, entwickelt unter Adobe ColdFusion.}
}
\newglossaryentry{ColdFusion}{
	name=ColdFusion,
	description={Applicationserver, welcher die Sprache CFML anbietet. Neben der kostenpflichtigen Version von Adobe existieren auch OpenSource Implementierungen, wie beispielsweise Railo.}
}