%!TEX root = ../document.tex
\chapter{Testergebnisse}
\label{ch:test_reults}


\renewcommand{\arraystretch}{1.2}
\arrayrulecolor{TableRuleColor}
\begin{table}[ht!]
\centering
\begin{tabular}{l | c | lllll}
Test & $\beta$ & Solr & Volltext & CloudSearch & LIKE &  Trigramme \\
\hline
\# 1 Volltextsuche – Effektivität & 5 & 0,38 & 0,4 & 0,4 & 0,59 & 0,46 \\
\# 2 Volltextsuche – Durchsatz & 2 & 0,33 & 1 & 0,11 & 0,26 & 0,01 \\
\# 3 Volltextsuche – Latenz & 3 & 0,9 & 0,9 & 0,7 & 0,8 & 0 \\
\# 4 Filterung nach Datum & 2 & 1 & 1 & 1 & 1 & 1 \\
\# 5 Filterung nach numerischer ID & 2 & 1 & 1 & 1 & 1 & 1 \\
\# 6 Umkreissuche & 1 & 1 & 0 & 1 & 0 & 0 \\
\# 7 Indizierungsgeschwindigkeit & 1 & 1 & 1 & 0 & 1 & 1 \\
\# 8 Mehrsprachige Inhalte & 1 & 1 & 0,95 & 1 & 1 & 1 \\
\# 9 Rechtschreibkorrektur & 2 & 0,78 & 0,17 & 0,9 & 0 & 0,97 \\
\# 10 Durchsuchung von Dateien & 1 & 0 & 0 & 0 & 0 & 0 \\
\# 11 Jährliche Kosten & 2 & 1 & 1 & 0,7 & 1 & 1 \\
\# 12 Suchbegriff-Highlighting & 1 & 1 & 1 & 1 & 0 & 0 \\
\# 13 Wildcards-Unterstützung & 0,2 & 1 & 0 & 0 & 1 & 0 \\
\# 14 Boolesche Operatoren & 0,2 & 1 & 1 & 1 & 0 & 0 \\
\hline
\hline
Summe & 23,4 & 17,42 & 15,99 & 14,72 & 14,07 & 12,26 \\
Platzierung &  & 1 & 2 & 3 & 4 & 5

\end{tabular}
\caption[Testresultate]{Resultate der Tests aus Anhang \ref{ch:test_Definition}. Volltext, LIKE und Trigramme beziehen sich auf die jeweiligen Suchfunktionen in PostgreSQL.}
\label{tab:testresults}
\end{table}


\renewcommand{\arraystretch}{1.2}
\arrayrulecolor{TableRuleColor}
\begin{table}[ht!]
\begin{subtable}[c]{0.45\textwidth}
\begin{tabular}{l | rr}
 & recall & precision \\
\hline
LIKE & 0,5 & 0,73 \\
Trigramme & 0,375 & 0,6 \\
Volltext & 0,25 & 1 \\
Solr & 0,25 & 0,8 \\
CloudSearch & 0,25 & 1 \\
\end{tabular}
\caption[Recall und Precision]{Recall und Precision}
\label{tab:rec_prec}
\end{subtable}
\begin{subtable}[c]{0.45\textwidth}
\begin{tabular}{l | rr}
 &  Latenz in $s$ & Durchsatz in $\frac{Anfragen}{s}$ \\
\hline
LIKE & 0,50 & 55,4 \\
Trigramme & 12,74 & 1,97 \\
Volltext & 0,32 & 212,3 \\
Solr & 0,31 & 69,3 \\
CloudSearch & 0,61 & 23,4 \\
\end{tabular}
\caption[Latenz und Durchsatz]{Latenz und Durchsatz}
\label{tab:latency_through}
\end{subtable}
\caption{Messwerte bei einzelnen Tests}
\end{table}