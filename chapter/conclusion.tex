%!TEX root = ../document.tex
\chapter{Fazit}

% Zusammenfassung der Ergebnisse, Abgabe einer Empfehlung, persönliches Fazit, \emph{Lessons Learned}

\section{Auswertung der Testergebnisse}

Die Ergebnisse der in Anhang \ref{ch:test_Definition} beschriebenen Tests werden in Tabelle \ref{tab:testresults} aufgeführt. Als Testsieger lässt sich Apache Solr ausmachen, welches sich mit knappen Vorsprung vor der in PostgreSQL integrierten Volltextsuche platziert hat. Wenn auch mit einem Hauch von Willkürlichkeit behaftet und dadurch in ihrer Aussagekraft eingeschränkt, so erlauben die Tests dennoch einen groben Einblick in die Stärken und Schwächen der einzelnen Systeme.

\paragraph{Trigramme} Überrascht hat so beispielsweise der Trigramm-basierte Ansatz: Obwohl überragend bei der Erkennung von Rechtschreib- und Tippfehlern sinkt die \emph{Performanz} mit steigender Anzahl der hinterlegten Datensätze stark. Tests ergaben, dass insbesondere das vergleichsweise lange Freitextfeld für die Veranstaltungsbeschreibung hierfür verantwortlich war: Lässt man dieses beim Trigramm-Vergleich außen vor, so reduziert sich die durchschnittliche Abfragedauer auf ähnliche Werte wie die der anderen Systeme.

\paragraph{LIKE-Suche} Ebenfalls erwähnenswert ist das hohe Resultat der \mintinline{sql}{LIKE}-basierten Suche für die \emph{Effektivität}. Dieses Ergebnis ist möglicherweise dem geringen Umfang und der niedrigen Qualität der Testdaten und -Anfragen geschuldet. Wie auch für die Trigramme gilt für die grundsätzlich SQL-basierte Suche, dass komplexere Filter-Funktionen nur mit Hilfe von Erweiterungen möglich sind, die in diesen Test nicht mit einbezogen wurden.

\paragraph{Amazon CloudSearch} Auch für eine Überraschung gut war das kostenpflichtige Produkt von Amazon: Erst gegen Ende der Evaluation stellte sich so heraus, dass Amazon CloudSearch selbst wiederum auf Apache Solr basiert.\footnote{siehe \url{http://aws.amazon.com/de/cloudsearch/faqs/}} Insbesondere die API für Suchabfragen ähnelt sich stark. Auch wenn sich die Kosten -- wie in Abschnitt \ref{sec:costs_amazon} kalkuliert -- in Grenzen halten, so rechnet sich eine cloudbasierte Lösung vermutlich erst bei höheren Lastanforderungen als für die Veranstaltungsverwaltung derzeit in Betracht kommen. In diesem Fall könnte es hilfreich sein, mit Apache Solr schon zuvor auf eine Lösung mit ähnlicher Schnittstelle zu setzen, um die Migration zu vereinfachen. Einen Nachteil von Amazon CloudSearch, der von den Tests nicht erfasst wurde, stellen die hohen Wartezeiten nach Konfigurationsänderungen dar, welche die Entwicklung einer Anbindung an diese Suche erschweren könnten. So ist das System nach einer Änderung des Dokumentenschemas erst nach zehn Minuten bis einer halben Stunde wieder voll einsatzbereit.

\paragraph{PostgreSQL -- Volltextsuche} In puncto \emph{Latenz und Durchsatz} punktete stattdessen die in PostgreSQL integrierte Volltextsuche. Lediglich bei der Erkennung von Rechtschreibfehlern schwächelte die Lösung -- was sich aber möglicherweise durch eine Kombination mit Trigrammen oder Levenshtein ausgleichen lässt.

\paragraph{Apache Solr} erweist sich schließlich der Punktzahl nach als Testsieger. Die bei Evaluation gewonnen Erfahrungen aus dem Umgang mit dieser Search Engine sowie in diesem Zusammenhang durchgeführte Recherchen lassen jedoch darauf schließen, dass Solr die bei Weitem umfangreichste der hier thematisierten Anwendungen ist. Ähnlich wie  PostgreSQL kann auch Solr erweitert werden: So kann die Suchmaschine mit \emph{Apache Tika}\footnote{vergleiche \url{https://tika.apache.org/1.11/formats.html}} kombiniert werden, um auch beliebige Dateien wie PDF, Word oder MP3 nach Textinhalten oder Metadaten zu durchsuchen. In Verbindung mit der freien OCR-Software \emph{tesseract} ermöglicht es die Indizierung von Bildinhalten.

\section{Empfehlung}

In Hinblick auf die Integration in ReEvent würde sich die in PostgreSQL eingebaute Volltextsuche als einfach umzusetzende Lösung empfehlen, welche bereits viele Leistungsmerkmale einer vollwertigen Suchmaschine mit sich bringt. Da der von ReEvent verwendete OR-Mapper Doctrine wie bereits in Abschnitt \ref{sec:reevent.database} dargestellt auch PostgreSQL unterstützt, wäre in diesem Fall nur eine einzelne Datenhaltung notwendig -- die Synchronisation einer zusätzlichen Datenbank wie beim Einsatz von Apache Solr würde entfallen. Auch für andere, einfache Anwendungen würde sich der Einsatz dieser Lösung aus Kostengründen empfehlen. Beispielsweise könnte man die Suche im Altsystem INsite für die restliche Dauer der Entwicklung von ReEvent hierauf umstellen, um bereits jetzt eine hochwertige Volltextsuche zu erhalten. Vor Einrichtung der PostgreSQL"=basierten Volltextsuche würde es sich empfehlen, die Indexnutzung nochmals mit einer Kopie von Live"=Datenbanken zu testen und zu überprüfen, ob nicht doch ein Datenbankindex eingerichtet werden sollte.\footnote{vergleiche hierzu die Diskussion zu Datenbankindizes in Abschnitt \ref{sec:eval_indices}, wobei angemerkt wurde, dass PostgreSQL eingerichtete Indizes scheinbar nicht nutzt}

Soll die Suche allerdings umfangreicher implementiert werden -- insbesondere in Hinblick auf externe Dateien oder Geodaten -- so wäre Apache Solr das Mittel der Wahl. Den höheren Funktionsumfang erkauft man sich allerdings mit erhöhten Konfigurations- und Entwicklungsaufwand. Vor allem die Konfiguration des Dokumentenschemas erfordert einiges an Zeit für die Einarbeitung, wie während den Testläufen aufgefallen ist. Dafür erhält man ein Suchsystem, welches nahezu alle Anforderungen aus Kapitel \ref{ch:requirements} erfüllen kann. Sollte die Menge der Anfragen oder die Anzahl der zu durchsuchenden Elemente stärker zunehmen als für die hier verwendeten Tests prognostiziert, so kann Apache Solr auch als verteilte Search Engine eingerichtet werden um die Last auf mehrere Knoten zu verteilen.

\section{Ausblick}

Unabhängig von den Testergebnissen lässt sich feststellen, dass auch in Zeiten von NoSQL"=Datenbanken die relationalen Vertreter durchaus eine Existenzberechtigung haben. Vor allem PostgreSQL konnte mit der integrierten Volltextsuche die eigene Funktionsvielfalt unterstreichen.

Auch wenn die Suchfunktion einer Anwendung meist eher in den Hintergrund tritt, so zeigt diese Arbeit dennoch, dass das Feld der Volltextsuche ein komplexes und umfangreiches ist. Die hier abgegebene Empfehlung erlaubt es nun eine Anbindung der gewählten Suchlösung an ReEvent zu implementieren. Ein möglicher Ansatz hierbei könnte die in Klassendefinitionen verwendeten Annotationen für Datentypen auswerten und Suchanfragen und Indizierungsanforderungen daraus generisch erstellen. So könnten möglicherweise die bereits vorhandenen Annotationen für Doctrine wiederverwendet werden, um eine \emph{schema.xml}"=Datei für Apache Solr zu erzeugen. Eine derart entwickelte Suchanbindung wäre somit nicht notwendigerweise spezifisch für ReEvent und würde daher auch eine Wiederverwendung in andere Flow-Applikationen erlauben. Möglicherweise könnte ein solches Modul als Open Source"=Veröffentlichung einen Grundstein für die weitere Verbreitung des Flow"=Frameworks darstellen.


