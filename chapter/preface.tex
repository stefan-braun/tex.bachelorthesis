%!TEX root = ../document.tex
\chapter*{Vorwort}
\addcontentsline{toc}{chapter}{\texorpdfstring{\spacedlowsmallcaps{Vorwort}}{Vorwort}}
\phantomsection

\begin{chapquote}{internet live stats \cite{InternetLiveStats.2014}}
	Im Jahr 2014 wurde eine Milliarde registrierter Internetseiten verzeichnet.\footnote{Gezählt wurden registrierte Domains, welche auch zu IP-Adressen auflösen \cite{InternetLiveStats.2014}}
\end{chapquote}

\vspace*{2\baselineskip}

Seit der Entstehung des Internets ist die Menge der Webseiten und der darüber abrufbaren Daten ins nahezu Unermessliche gestiegen. Lediglich mit Hilfe von Websuchmaschinen wie Google, Yahoo oder Bing ist es Nutzern noch möglich, aus dieser Informationsflut die für sie relevanten Inhalte aufzuspüren.

Doch nicht nur im Web nimmt die Größe von Datensätzen stetig zu: In Zeiten von BigData und umfangreichen lokalen Applikationen gestaltet es sich auch im Unternehmens\-umfeld immer schwieriger, den Überblick zu behalten. Wie für das Internet die Websuchmaschinen, so erlauben lokale \emph{Search Engines} die Durchsuchung interner Dokumente und Datenbanken. Und wie im Internet bei der Auswahl der Websuchmaschine steht auch hier der Anwender vor der Wahl: Welche \emph{Search Engine} kann meine Anforderungen am besten erfüllen?

\vspace*{3\baselineskip}
\noindent
{\normalfont\Large\sffamily  \spacedallcaps{Danksagung}}
\vspace*{2\baselineskip}

\noindent
\textbf{allen Freunden, Kommilitonen und meiner Familie} für die Unterstützung während des gesamten Studiums und der Korrekturlesung,
\vspace*{5pt}

\noindent
\textbf{Prof. Dr. Regensburger} von der TH Ingolstadt für die Betreuung dieser Arbeit,
\vspace*{5pt}

\noindent
\textbf{Günter Huber und Gerhard Rupp} von response für die Vergabe des Bachelorarbeitsthemas,
\vspace*{5pt}

\noindent
meinen \textbf{Arbeitskollegen} bei response für die Hilfe rund um technische Aspekte von ReEvent, Neos CMS und Flow,
\vspace*{5pt}

\noindent
\textbf{Dr. Paul Spannaus und David Dionis} von der TH Ingolstadt für ihre Einführung zu \ctLaTeX\,aus dem Kurs \enquote{Professionelle Textsatzsysteme},
\vspace*{5pt}

\noindent
der \textbf{tex.stackexchange.com} Community für ihre Hilfe in vielen kleineren und auch größeren Problemen rund um das Thema Textsatz,
\vspace*{5pt}

\noindent
sowie \textbf{allen anderen}, die der Meinung sind, vergessen worden zu sein -- Danke!

