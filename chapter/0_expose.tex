%!TEX root = ../document.tex

\chapter{Themenübersicht und Motivation}

Öffentliche Veranstaltungen, Rathaussitzungen sowie Stadtführungen der Stadt~Ingolstadt werden derzeit über das webbasierte Veranstaltungsverwaltungssystem \emph{INsite} publik gemacht. Aufgrund technischer, funktionaler sowie softwareergonomischer Mängel soll \emph{INsite} durch eine Neuentwicklung ersetzt werden. Das neue System namens \emph{ReEvent} wird ebenso wie \emph{INsite} von der response informationsdesign~gmbh~\&~co.~kg entwickelt und basiert --- statt auf Adobe ColdFusion bei \emph{INsite} --- auf dem PHP"=Framework Flow und dem Content Management System Neos.

Die circa 3000~Veranstaltungen jährlich, welche bis dato in \emph{INsite} größtenteils händisch eingetragen werden, können über ein Formular auf der Website der Stadt~Ingolstadt gefiltert und durchsucht werden. Die Suche wird realisiert durch eine SQL"=Abfrage, welche Name und Beschreibung der Veranstaltung nach einem gegebenen String durchsucht. Gegenstand dieser Arbeit ist die Evaluation diverser Möglichkeiten, eine Suchlösung für das neue System \emph{ReEvent} zu implementieren.

Dem zu Grunde liegend ist eine Anforderungsanalyse: Im Gespräch mit den Nutzern der Software sowie dem Projektmanagement bei response muss eruiert werden, welche Funktionen das Suchsystem insbesondere im Vergleich zur bestehenden Suche bieten soll. Denkenswert wären Featurewünsche, wie beispielsweise automatische Vervollständigung von Suchvorschlägen, Rechtschreibkorrektur von Suchbegriffen und das Erzeugen komplexer Suchabfragen. Zudem muss die Suche grundsätzlich in Lage sein, die vom verwendeten Framework abgespeicherten Daten zu indizieren und zu durchsuchen. Da für das Datenmodell von \emph{ReEvent} ein gemischter Ansatz verfolgt wird --- teils dokumentenorientiert und teils objektorientiert --- müssen die Daten für die Suche möglicherweise aufbereitet werden. Weitere Anforderungen könnten durch den Systemkontext um \emph{ReEvent} und weitere Webserver der Stadt Ingolstadt entstehen. Es stellt sich ebenfalls die Frage, inwiefern Flexibilität und Verteilbarkeit der Lösung eine Rolle spielen.

Auf Basis der gefundenen Anforderungen kann im Anschluss ein Kriterienkatalog aufgestellt werden, anhand dessen einzelne Lösungen systematisch evaluiert werden können. Des Weiteren muss eine Testumgebung aufgebaut werden, um realitätsnahe Ergebnisse für beispielsweise Performancetests oder Tests der Suchergebnisse zu erhalten. Als Testdaten können hier die bestehenden Datensätze aus der \emph{INsite}-Datenbank dienen, welche derzeit etwa 50000~Veranstaltungen umfasst. Gegebenenfalls kann es sinnvoll sein, für Lasttests größere Datenmengen zu erzeugen. Zusätzlich zu Performance könnte auch die Qualität der Suchergebnisse eine Rolle spielen.

Zu Beginn der Arbeit muss außerdem eine begründete Vorauswahl von Suchsystemen getroffen werden. Wahrscheinliche Kandidaten sind an dieser Stelle \emph{Apache Solr} und \emph{ElasticSearch}, welche beide auf der JAVA"=Bibliothek \emph{Apache Lucene} aufsetzen. \emph{Apache Solr} empfiehlt sich zudem dahingehend, dass es indirekt bereits von response für Teilbereiche des städtischen Internetauftrittes genutzt wird. Somit könnte die Einarbeitungszeit für \emph{Apache Solr} vergleichsweise geringer sein, was wiederum ein mögliches Kriterium des Entscheidungsprozesses darstellt. \emph{ElasticSearch} als direkter Konkurrent zu \emph{Apache Solr} wirbt wiederum mit guter Skalierbarkeit. Eine weitere Lösung stellt die \emph{Google Search Appliance} dar, welche als dedizierte Hardware prinzipiell die Suchfunktionalität von Google anbietet. Zudem können auch cloudbasierte Varianten untersucht werden, wie beispielsweise \emph{Microsoft AzureSearch} und \emph{Amazon CloudSearch}. Zusätzlich sollte auch ein Vergleich mit der bestehenden Standardsuche von Neos gezogen werden.

Um die einzelnen Lösungen mit Neos / Flow als Zielsystem evaluieren zu können, wird es für die Indizierung notwendig sein eine Schnittstelle zwischen Suchserver und Neos zu implementieren. Eine entsprechende Implementierung zu ElasticSearch existiert bereits. Anspruch an die Implementierungen könnte sein, die Anpassung an ein gezieltes Suchsystem möglichst abstrakt zu gestalten um den Implementierungsaufwand pro Suchsystem gering zu halten. Abschließend sollten es die durchgeführten Tests und Recherchen erlauben, ein Suchsystem für die neue Veranstaltungsverwaltung auszuwählen.

Des Weiteren kann zur Einordnung des Themas kurz auf response als auftraggebende Firma, sowie auf die Stadt Ingolstadt als Kunden eingegangen werden. Weitere Punkte wären an dieser Stelle eine Vorstellung der Systeme \emph{INsite}, \emph{ReEvent}, Flow und Neos.

Eingangs könnte zudem eine theoretische Abhandlung von für Suchsysteme relevanten Algorithmen erstellt werden. Mögliche Bestandteile wären Indizierung, Stemming und Normalisierung. Raum für eine Ausdehnung des Themas bietet die Entwicklung eines softwareergonomischen Nutzerinterfaces für komplexere Funktionen der Suche, spezialisiert auf das ausgewählte Suchsystem und den Veranstaltungskalender \emph{ReEvent}. Einfache Beispiele für derartige Funktionen stellen logische Verknüpfungsoperatoren zwischen Suchbegriffen dar.